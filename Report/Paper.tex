%\documentclass[a4paper,english,12pt,twocolumn]{article}
\documentclass[a4paper,english,12pt]{article}
\usepackage[utf8]{inputenc} % Encodage du fichier
\usepackage[T1]{fontenc} % Encodage des fonts nécessaire pour le Latin
\usepackage[french]{babel} % Pour changer la langue des mots générés et choisir la bonne mise en page
\usepackage{amssymb}
\usepackage{pdflscape}
\usepackage{microtype} 
\usepackage{lmodern} % Le latin modèrne
\usepackage[top=2cm, bottom=2cm, left=2.5cm, right=1.5cm]{geometry} % Définir les marges de la page 
\usepackage[hidelinks,urlcolor=blue,unicode=true,
pdftitle={Feed Forward Neural Network with Back Propagation MNIST},
pdfauthor={BELOUADAH Eden},
pdfdisplaydoctitle=true]{hyperref} % Pour les liens 
\usepackage{fancyhdr} % Pour le style de la page
\usepackage[font=it]{caption} % Rendre les titres des tableaux italiques
\usepackage{graphicx} % Pour les images
\usepackage{subcaption} % Pour mettre plusieurs images sur la même ligne
\usepackage{float} % Pour empêcher le déplacement des tableaux et des figures.
\usepackage{babelbib} % Pour changer la langue dans la bibliographie
\usepackage{amsmath} % Pour des fonctions mathématiques
\usepackage{amssymb} % Pour les symboles mathématiques
%\usepackage[onelanguage,english,longend,boxruled,algoruled,linesnumbered,algochapter,nofillcomment]{algorithm2e} %pour les algorithmes
\usepackage{multirow}
\usepackage{booktabs}
\usepackage{enumitem}
\usepackage{setspace}

\graphicspath{ {images/} } % Spécifier le répertoire contenant les images

\DisableLigatures[f]{encoding=*}

%Active ça si tu ne veux pas les points-virgules dans les algorithmes
% \DontPrintSemicolon
 
%\renewcommand \thechapter{\Roman{chapter}} % Utiliser les numéros romans pour les chapitres

\captionsetup{labelfont=it,textfont=it,labelsep=period} % Changer le style des légendes
\AtBeginDocument{ % Changer les légendes
	\renewcommand\tablename{\itshape Tableau}
	\renewcommand{\figurename}{\itshape Figure}
	% Renommer la table des matières
	\renewcommand{\contentsname}{Sommaire}
}

% Style de l'entête et le pied de la page
\setlength{\headheight}{16pt}
\pagestyle{fancy}
\fancyhead[L]{} % Enlever la section
\fancyhead[R]{\footnotesize\slshape{\nouppercase{\leftmark}}} % Titre du chapitre en minuscule avec taille 10
\fancyfoot[C]{}
\fancyfoot[R]{\thepage} % Déplacer le numéro de la page vers la droite de la page

\fancypagestyle{plain}{
\renewcommand{\headrulewidth}{0pt}
\fancyhf{}
\fancyfoot[R]{\thepage}
}
  
% Espace entre les lignes
\linespread{1.3}

% Code pris de https://tex.stackexchange.com/a/95616/109916 et corrigé
% Début
\makeatletter
\newcommand{\emptypage}[1]{
  \cleardoublepage
  \begingroup
  \let\ps@plain\ps@empty
  \pagestyle{empty}
  #1
  \cleardoublepage
  \endgroup}
\makeatletter
% Fin


% pour changer les deux points des légendes d'algorithmes
% \SetAlgoCaptionSeparator{\unskip.}

\begin{document}
%\include{Page_de_garde}
%\include{Remerciements}
\emptypage{
%\tableofcontents
%\listoffigures
%\listoftables
}
    
\setlength{\parskip}{0.6em plus 0.1em minus 0.1em}
%\SetKwInput{KwOut}{Outpits}

% Redéfinition des chapitres et sections pour les inclure dans le sommaire
\makeatletter
%	\let\oldchapter\chapter
%	\newcommand{\@chapterstar}[1]{\cleardoublepage\phantomsection\addcontentsline{toc}{chapter}{#1}{\oldchapter*{#1}}\markboth{#1}{}}
%	\newcommand{\@chapternostar}[1]{{\oldchapter{#1}}}
%	\renewcommand{\chapter}{\@ifstar{\@chapterstar}{\@chapternostar}}
\let\oldsection\section
\newcommand{\@sectionstar}[1]{\phantomsection\addcontentsline{toc}{section}{#1}{\oldsection*{#1}}}
\newcommand{\@sectionnostar}[1]{{\oldsection{#1}}}
\renewcommand\section{\@ifstar{\@sectionstar}{\@sectionnostar}}	
\newcommand*{\rom}[1]{\expandafter\@slowromancap\romannumeral #1@}
\makeatother

\setcounter{page}{1}
%%%%%%%%%%%%%%%%%%%%%%%%%%%%%%%%%%%%%%%%%%%%%%%%%%%%%%%%%%%%%

\title{Feed Forward Neural Network with Back Propagation \\ Application on MNIST Dataset}
\author{Eden BELOUADAH}
% \date{}
\maketitle

\section{Introduction}
Le domaine d'apprentissage profond est l'un des domaines les plus récents qui s'inspirent du domaine biomédical. Il trouve des applications dans plein de problématiques réelles parmi lesquelles la classification d'image. 

Le but de ce travail pratique est de mettre en oeuvre un réseau de neurones Feed Forward avec la rétropropagation de l'erreur, et l'appliquer sur la base de données MNIST.

La base de données MNIST est contruite de 50000 exemple d'images binaires numérisées. Chaque image contient un chiffre manuscrit et le but du réseau est d'associer l'image au bon chiffre correspondant parmi 10. 


\section{Meilleure architecture du réseau de neurones}
La meilleure architecture obtenue pour ce réseau de neurone est la suivante: 
\begin{itemize}
	\item Nombre de couches cachées: 1,
	\item Nombre de neurones dans chaque couche cachée: 20,
	\item Fonction d'activation : ReLU,
	\item Erreur: Quadratique.
\end{itemize}

\section{Analyse des résultats}
Nous remarquons que lorsque les dérivées sont très petites, les deux fonctions $sigmoid$ et $tanh$ ne deviennent plus efficaces et l'effet du gradient disparait. Donc la fonction ReLU reste le meilleur choix pour ce problème.


\begin{figure}[H]
	\centering
	\includegraphics[width=0.8\textwidth,keepaspectratio]{Error}
	\caption{Sample image\label{img1}}
\end{figure}

\begin{table}[H]\centering
	\begin{tabular}{ccccccc}
		\toprule \textbf{Architecture} & 1 & 2 & 3 & 4 & 5\\    \midrule
		\textbf{Taux de précision} & 100\% & 97.5\% & 92.06\% & 89.66\% & 89.19\%  
		\\    \midrule
		\textbf{Taux d'erreur} & 100\% & 97.5\% & 92.06\% & 89.66\% & 89.19\% \\   
		\bottomrule	
	\end{tabular}
	\caption{Précision et Erreur en fonction d'architecture\label{tab1}}
\end{table}

\begin{itemize}
	\item Architecture1:,
	\item Architecture2:,
	\item Architecture3:,
	\item Architecture4:.
\end{itemize}

\section{Conclusion}
Ce travail pratique était une occasion pour voir de plus proche le fonctionnement des réseaux de neurones. Ces derniers peuvent changer d'architecture en variant le type du problème, sa difficulté, ses particularités... 

La base de données MNIST représente une tâche de classification simple qui peut être résolue avec un réseau de neurone simple, même sans couches cachées, comme nous l'avons vu dans le travail pratique précédent. En effet, il était possible d'atteindre une précision de 87\% facilement et en peu de temps.

Les architectures complexes sont plus adaptées aux problèmes complexes qui nécessitent vraiment un réseau profond.
%%%%%%%%%%%%%%%%%%%%%%%%%%%%%%%%%%%%%%%%%%%%%%%%%%%%%%%%%%%%%
\bibliographystyle{babplain}
\parskip=-1em
%\emptypage{\bibliography{bibliographie}}
\let\section\oldsection % pour éviter que le résumé soient visibles dans le sommaire comme une section
%\include{Resume}
\end{document}
